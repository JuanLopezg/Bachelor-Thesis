\chapter{Introducción}
\label{ch:intro}

Al estudiar topología, una de las cuestiones más relevantes es si dos espacios son ``topológicamente'' equivalentes, es decir, si existe un homeomorfismo entre ellos. Si existe un homeomorfismo entre dos espacios topológicos, podemos demostrar que comparten las mismas propiedades. Sin embargo, encontrar homeomorfismos o demostrar que no existe un homeomorfismo entre dos espacios es una tarea difícil. Por ejemplo, los espacios $\mathbb{R}$ y $\mathbb{R} \setminus \{0\}$ son fácilmente diferenciables por el concepto de conexión. Mientras que $\mathbb{R}$ es conexo, el segundo deja de serlo al quitarle el punto $\{0\}$. En cambio, si queremos diferenciar $\mathbb{R}^2$ de $\mathbb{R}^2 \setminus \{(0,0)\}$, no podemos atender a la topología general con el concepto de conexión. Aquí es donde entra el concepto del grupo fundamental (topología algebraica). Este es un grupo de homotopía que contiene las clases de equivalencia de lazos, o aplicaciones $f \colon S^1 \to X$, salvo homotopía. De esta forma podemos establecer que $\mathbb{R}^2$ y $\mathbb{R}^2 \setminus \{(0,0)\}$ no son homeomorfos al no tener grupos fundamentales isomorfos.

Los grupos fundamentales proporcionan una estructura algebraica adicional a la topología. Si dos espacios son topológicamente equivalentes, entonces los espacios tienen grupos fundamentales isomorfos. Supongamos que $X$ es un espacio topológico. Un recubrimiento de $X$ es un espacio $E$ con una aplicación continua sobre $X$ que satisface una condición muy fuerte. Como veremos posteriormente esta condición implica que $E$ estará formado por diferentes ''hojas'' o ''capas'' que se aplican homeomórficamente por esa aplicación, quedando $X$ totalmente ''cubierto''. Como primer ejemplo, piensa en ``envolver'' la recta real alrededor de un círculo. El grupo fundamental de un espacio está estrechamente relacionado con su espacio recubridor. Además, los subgrupos del grupo fundamental de un espacio topológico $X$ nos ayudan a clasificar todos sus espacios recubridores.

Un espacio topológico $X$ puede tener varios recubridores, pero ¿cómo podemos diferenciarlos o determinarlos todos si no los conocemos?. Para resolver esas cuestiones veremos que podremos utilizar los subgrupos del grupo fundamental del espacio recubierto $X$. Además, si se cumplen ciertas condiciones en $X$, podremos afirmar la existencia de un espacio capaz de recubrir a cualquier otro espacio recubridor. Por esta razón, este espacio recibirá el nombre de ''espacio recubridor universal''.

Hasta ahora, hemos estado viendo cómo se aplica la teoría algebraica al estudio de estos espacios recubridores. Pero también podemos hacer un estudio en el sentido contrario. Los grafos son estructuras algebraicas formadas por un conjunto de vértices y otro de aristas, que conectan pares de vértices. Estos son herramientas muy importantes tanto en matemáticas como en informática y otras disciplinas. La razón yace en su utilización para modelar y analizar redes, en algoritmos de búsqueda, problemas de optimización combinatoria como el problema del viajante,.. En este trabajo utilizaremos la teoría de espacios recubridores para poder estudiar estas estructuras, resolviendo un problema de combinatoria en el cálculo de su grupo fundamental y analizando algunas de sus propiedades.

El propósito de este Trabajo de Fin de Grado (TFG) es el estudio de los espacios recubridores y de sus aplicaciones algebraicas. Veremos que, al analizar estos recubrimientos, comprendemos mejor la estructura topológica de los espacios y cómo se relacionan entre sí. 
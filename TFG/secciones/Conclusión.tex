\chapter{Conclusión}

En este trabajo hemos estudiado los espacios recubridores en topología y algunas de sus aplicaciones algebraicas. Comenzamos comprendiendo lo que es una aplicación recubridora y su espacio recubridor asociado, viendo algunos ejemplos de ambos y centrádonos en el estudio del grupo fundamental del círculo. Para ello, recordamos los conceptos de caminos y homotopías y aplicamos el término del levantamiento.

Posteriormente, vimos el término de equivalencia de aplicaciones recubridoras para entender el concepto del espacio recubridor universal. Además, estudiamos el concepto de transformaciones recubridoras su isomorfismo.

Finalmente, aplicamos la teoría de espacios recubridores aprendida a la teoría de grupos, centrándonos en el estudio de grafos finitos y conexos. Conseguimos probar algunas de sus propiedades y demostramos el teorema que determinaba el cardinal de un grupo de generadores libres del grupo fiundamental de un grafo. Además, iniciamos un breve estudio sobre el concepto del número de Euler.

Una continuación futura pordría considerar el estudio de los subgrupos de grupos libres. De esta forma podríamos establecer alguna relación entre el número de Euler y el cardinal de un sistema de generadores libres para el grupo fundamental de un grafo.
\chapter*{Resumen}

Este Trabajo de Fin de Grado surge como una continuación natural de la asignatura de Topología, explorando a fondo el concepto de los espacios recubridores y sus aplicaciones algebraicas. 

La estructura del TFG sigue la organización de los capítulos 13 y 14 del libro "Topología" de James R. Munkres \cite{Topología} , pero con un primer capítulo que se centra en una introducción detallada, abordando los conceptos básicos de espacios recubridores, aplicaciones recubridoras y levantamiento de caminos, según las secciones 53 y 54 del capítulo 9 del libro. Esta sección servirá como base para el estudio más profundo que se realizará en los siguientes capítulos. Es importante recalcar también que se han utilizado como libros de apoyo ''Topología'' de Héctor Barge Yáñez y Alfonso Zamora Saiz \cite{Alfonso}, y ''Algebraic Topology'' de Allen Hatcher \cite{Allen}.

En la segunda parte, el lector se sumergirá en la clasificación de espacios recubridores y en el estudio de un teorema crucial que asegura la existencia y unicidad de un espacio recubridor universal. Además, nos detendremos en el análisis del recubridor universal de diversos espacios, y finalizaremos el capítulo con el concepto de transformación recubridora.

La última parte del TFG se enfocará en las aplicaciones algebraicas de los espacios recubridores, centrándose en la teoría de grupos. Se explorarán los grafos finitos como espacios topológicos, estudiando sus recubridores y el grupo fundamental asociado.

En resumen, este TFG ofrece una exploración exhaustiva de los espacios recubridores, proporcionando una base sólida para comprender su clasificación, propiedades y aplicaciones algebraicas, especialmente en el contexto de la teoría de grupos.



\begin{otherlanguage}{english}
  \chapter*{Abstract}

This Final Degree Project arises as a natural continuation of the Topology subject, exploring in depth the concept of covering spaces and their algebraic applications.

The structure of the TFG follows the organization of chapters 13 and 14 of the book "Topology" by James R. Munkres \cite{Topología} , but with a first chapter that focuses on a detailed introduction, addressing the basic concepts of covering spaces, covering maps and path lifting, according to sections 53 and 54 of chapter 9 of the book. This section will serve as the basis for the more in-depth study that will be carried out in the following chapters. Last but not least, ''Topología'' by Héctor Barge Yáñez and Alfonso Zamora Saiz \cite{Alfonso}, and ''Algebraic Topology'' by Allen Hatcher \cite{Allen}, have been used as support books.

In the second part, the reader will dive into the classification of covering spaces and the study of a crucial theorem that ensures the existence and uniqueness of a universal covering space. Furthermore, we will stop at the analysis of some universal covering spaces, and we will end the chapter with the concept of covering transformation.

The last part of the TFG will focus on the algebraic applications of covering spaces, focusing on group theory. Finite graphs will be explored as topological spaces, studying their covers and the associated fundamental group.

In summary, this TFG offers a comprehensive exploration of covering spaces, providing a solid foundation for understanding their classification, properties and algebraic applications, especially in the context of group theory.

\end{otherlanguage}


%%%%%%%%%%%%%%%%%%%%%%%%%%%%%%%%%%%%%%%%%%%%%%%%%%%%%%%%%%%
%% Final del resumen.
%%%%%%%%%%%%%%%%%%%%%%%%%%%%%%%%%%%%%%%%%%%%%%%%%%%%%%%%%%%
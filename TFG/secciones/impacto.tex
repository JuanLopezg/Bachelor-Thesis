\chapter{Análisis de impacto}

Los resultados probados en este Trabajo de Fina de Carrera no son ningún descubrimiento ya que se pueden encontrar tanto en el propio libro Topología de James R. Munkres \cite{Topología}  como en otras fuentes. Por ello, me centraré únicamente en el impacto que ha tenido este trabajo en mí.

Elegí hacer el trabajo sobre el ámbito de la Topología ya que la asignatura de la carrera del grado en Matemáticas e Informática captó mi interés. Parecía un campo de las matemáticas muy abstracto y difícil de comprender, pero que, una vez entendidos los conceptos se podría describir incluso como ''bonito''.

Esta ha sido la primera vez que hago un trabajo tan denso, resultándome igual de complicado que gratificante. Aunque muchos de los resultados se han obtenido del libro mencionado anteriormente, he logrado entenderlos e intentado explicarlos de una forma diferente para la comprensión del lector. Además, con este mismo objetivo he ido completando estos resultados con ejemplos sobre ellos. Con todo ello, he ampliado mis conocimientos en topología aún más, logrando una comprensión de los espacios recubridores, un concepto que era prácticamente nuevo para mí antes de comenzar el trabajo. Como efectos colaterales, he desarrollado mi habilidad para escribir pruebas matemáticas y, además, las he escrito en un lenguaje de programación totalmente nuevo para mí, TeX.
